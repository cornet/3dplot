\section{convert.h File Reference}
\label{convert_8h}\index{convert.h@{convert.h}}
Routines for converting plot data.  


\subsection*{Defines}
\begin{CompactItemize}
\item 
\#define {\bf STRIPWIDTH}\ 0.1
\end{CompactItemize}
\subsection*{Functions}
\begin{CompactItemize}
\item 
{\bf dataset} $\ast$ {\bf func\-To\-Para} (struct {\bf dataset} $\ast$ds)
\item 
{\bf dataset} $\ast$ {\bf kurv\-To\-Para} (struct {\bf dataset} $\ast$ds)
\item 
{\bf dataset} $\ast$ {\bf convert\-To\-Para} (char type[5], struct {\bf dataset} $\ast$ds)
\end{CompactItemize}


\subsection{Detailed Description}
Routines for converting plot data. 

3dplot deals with parametric data (PARA). Other formats are supported by 3dplot but are converted to parametric. The following data types are supported:\begin{itemize}
\item PARA Parametric data, this is the default\item FUNC Function plot data\item KURV Kurv plot data See the FILE\_\-SPEC doc for full details\end{itemize}


Definition in file {\bf convert.h}.

\subsection{Define Documentation}
\index{convert.h@{convert.h}!STRIPWIDTH@{STRIPWIDTH}}
\index{STRIPWIDTH@{STRIPWIDTH}!convert.h@{convert.h}}
\subsubsection{\setlength{\rightskip}{0pt plus 5cm}\#define STRIPWIDTH\ 0.1}\label{convert_8h_a0}


Width of line used to plot KURV Plots 

Definition at line 21 of file convert.h.

\subsection{Function Documentation}
\index{convert.h@{convert.h}!convertToPara@{convertToPara}}
\index{convertToPara@{convertToPara}!convert.h@{convert.h}}
\subsubsection{\setlength{\rightskip}{0pt plus 5cm}struct {\bf dataset}$\ast$ convert\-To\-Para (char {\em type}[5], struct {\bf dataset} $\ast$ {\em ds})}\label{convert_8h_a3}


Converting data to parametric

This is the function that should be called. It detects what type the data is and then calls one of specific converting functions.

\begin{Desc}
\item[Parameters:]
\begin{description}
\item[{\em type}]Sting containing plot type as defined in the data file 5 chars includs NULL terminator \item[{\em ds}]Pointer to data set to convert \end{description}
\end{Desc}
\begin{Desc}
\item[Returns:]Pointer to converted dataset \end{Desc}


Definition at line 84 of file convert.c.\index{convert.h@{convert.h}!funcToPara@{funcToPara}}
\index{funcToPara@{funcToPara}!convert.h@{convert.h}}
\subsubsection{\setlength{\rightskip}{0pt plus 5cm}struct {\bf dataset}$\ast$ func\-To\-Para (struct {\bf dataset} $\ast$ {\em ds})}\label{convert_8h_a1}


Converts Function data to Parametric

\begin{Desc}
\item[Parameters:]
\begin{description}
\item[{\em ds}]Pointer to data set to convert \end{description}
\end{Desc}
\begin{Desc}
\item[Returns:]Pointer to converted dataset \end{Desc}


Definition at line 14 of file convert.c.\index{convert.h@{convert.h}!kurvToPara@{kurvToPara}}
\index{kurvToPara@{kurvToPara}!convert.h@{convert.h}}
\subsubsection{\setlength{\rightskip}{0pt plus 5cm}struct {\bf dataset}$\ast$ kurv\-To\-Para (struct {\bf dataset} $\ast$ {\em ds})}\label{convert_8h_a2}


Converts Kurv (Curve) data to Parametric

\begin{Desc}
\item[Parameters:]
\begin{description}
\item[{\em ds}]Pointer to data set to convert \end{description}
\end{Desc}
\begin{Desc}
\item[Returns:]Pointer to converted dataset \end{Desc}


Definition at line 39 of file convert.c.