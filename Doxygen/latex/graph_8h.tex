\section{graph.h File Reference}
\label{graph_8h}\index{graph.h@{graph.h}}
Routines for plotting the graph itself.  


{\tt \#include \char`\"{}main.h\char`\"{}}\par
\subsection*{Functions}
\begin{CompactItemize}
\item 
void {\bf define\-Material} (void)
\item 
float {\bf red\-Map} (float height)
\item 
float {\bf green\-Map} (float height)
\item 
float {\bf blue\-Map} (float height)
\item 
void {\bf plot\-Graph} (struct {\bf dataset} $\ast$ds)
\end{CompactItemize}


\subsection{Detailed Description}
Routines for plotting the graph itself. 



Definition in file {\bf graph.h}.

\subsection{Function Documentation}
\index{graph.h@{graph.h}!blueMap@{blueMap}}
\index{blueMap@{blueMap}!graph.h@{graph.h}}
\subsubsection{\setlength{\rightskip}{0pt plus 5cm}float blue\-Map (float {\em height})}\label{graph_8h_a3}


Computers the Blue value for a height on the graph

This function assumes the height to be between -1 and 1 \begin{Desc}
\item[Parameters:]
\begin{description}
\item[{\em height}]Height of point. \end{description}
\end{Desc}
\begin{Desc}
\item[Returns:]Blue colour value as used by Open\-GL (i.e. between 0.0 and 1.0) \end{Desc}


Definition at line 43 of file graph.c.\index{graph.h@{graph.h}!defineMaterial@{defineMaterial}}
\index{defineMaterial@{defineMaterial}!graph.h@{graph.h}}
\subsubsection{\setlength{\rightskip}{0pt plus 5cm}void define\-Material (void)}\label{graph_8h_a0}


Defines the material used on the graph 

Definition at line 17 of file graph.c.\index{graph.h@{graph.h}!greenMap@{greenMap}}
\index{greenMap@{greenMap}!graph.h@{graph.h}}
\subsubsection{\setlength{\rightskip}{0pt plus 5cm}float green\-Map (float {\em height})}\label{graph_8h_a2}


Computers the Green value for a height on the graph

This function assumes the height to be between -1 and 1 \begin{Desc}
\item[Parameters:]
\begin{description}
\item[{\em height}]Height of point. \end{description}
\end{Desc}
\begin{Desc}
\item[Returns:]Green colour value as used by Open\-GL (i.e. between 0.0 and 1.0) \end{Desc}


Definition at line 36 of file graph.c.\index{graph.h@{graph.h}!plotGraph@{plotGraph}}
\index{plotGraph@{plotGraph}!graph.h@{graph.h}}
\subsubsection{\setlength{\rightskip}{0pt plus 5cm}void plot\-Graph (struct {\bf dataset} $\ast$ {\em ds})}\label{graph_8h_a4}


Plots the graph itself

This parses the dataset and plots the point. It assumes that the dataset is scaled so that all points lie between -1 and 1

\begin{Desc}
\item[Parameters:]
\begin{description}
\item[{\em ds}]dataset to plot \end{description}
\end{Desc}


Definition at line 50 of file graph.c.\index{graph.h@{graph.h}!redMap@{redMap}}
\index{redMap@{redMap}!graph.h@{graph.h}}
\subsubsection{\setlength{\rightskip}{0pt plus 5cm}float red\-Map (float {\em height})}\label{graph_8h_a1}


Computers the Red value for a height on the graph

This function assumes the height to be between -1 and 1 \begin{Desc}
\item[Parameters:]
\begin{description}
\item[{\em height}]Height of point. \end{description}
\end{Desc}
\begin{Desc}
\item[Returns:]Red colour value as used by Open\-GL (i.e. between 0.0 and 1.0) \end{Desc}


Definition at line 28 of file graph.c.