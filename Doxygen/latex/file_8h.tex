\section{file.h File Reference}
\label{file_8h}\index{file.h@{file.h}}
Routines for dealing with data files.  


{\tt \#include \char`\"{}main.h\char`\"{}}\par
\subsection*{Functions}
\begin{CompactItemize}
\item 
long {\bf getx\-Num} (struct {\bf dataset} $\ast$ds)
\item 
long {\bf gety\-Num} (struct {\bf dataset} $\ast$ds)
\item 
long {\bf getz\-Num} (struct {\bf dataset} $\ast$ds)
\item 
void {\bf destroy\-Data\-Set} (struct {\bf dataset} $\ast$ds)
\item 
{\bf dataset} $\ast$ {\bf read\-Data\-File} (FILE $\ast$fh)
\end{CompactItemize}


\subsection{Detailed Description}
Routines for dealing with data files. 



Definition in file {\bf file.h}.

\subsection{Function Documentation}
\index{file.h@{file.h}!destroyDataSet@{destroyDataSet}}
\index{destroyDataSet@{destroyDataSet}!file.h@{file.h}}
\subsubsection{\setlength{\rightskip}{0pt plus 5cm}void destroy\-Data\-Set (struct {\bf dataset} $\ast$ {\em ds})}\label{file_8h_a3}


Frees up memory used by unused dataset

\begin{Desc}
\item[Parameters:]
\begin{description}
\item[{\em ds}]Dataset to free up \end{description}
\end{Desc}


Definition at line 112 of file file.c.\index{file.h@{file.h}!getxNum@{getxNum}}
\index{getxNum@{getxNum}!file.h@{file.h}}
\subsubsection{\setlength{\rightskip}{0pt plus 5cm}long getx\-Num (struct {\bf dataset} $\ast$ {\em ds})}\label{file_8h_a0}


Gets the number of X values in the dataset

\begin{Desc}
\item[Parameters:]
\begin{description}
\item[{\em ds}]Pointer to dataset \end{description}
\end{Desc}
\begin{Desc}
\item[Returns:]Number of values. \end{Desc}


Definition at line 18 of file file.c.\index{file.h@{file.h}!getyNum@{getyNum}}
\index{getyNum@{getyNum}!file.h@{file.h}}
\subsubsection{\setlength{\rightskip}{0pt plus 5cm}long gety\-Num (struct {\bf dataset} $\ast$ {\em ds})}\label{file_8h_a1}


Gets the number of Y values in the dataset

\begin{Desc}
\item[Parameters:]
\begin{description}
\item[{\em ds}]Pointer to dataset \end{description}
\end{Desc}
\begin{Desc}
\item[Returns:]Number of values. \end{Desc}


Definition at line 23 of file file.c.\index{file.h@{file.h}!getzNum@{getzNum}}
\index{getzNum@{getzNum}!file.h@{file.h}}
\subsubsection{\setlength{\rightskip}{0pt plus 5cm}long getz\-Num (struct {\bf dataset} $\ast$ {\em ds})}\label{file_8h_a2}


Gets the number of Z values in the dataset

\begin{Desc}
\item[Parameters:]
\begin{description}
\item[{\em ds}]Pointer to dataset \end{description}
\end{Desc}
\begin{Desc}
\item[Returns:]Number of values. \end{Desc}


Definition at line 28 of file file.c.\index{file.h@{file.h}!readDataFile@{readDataFile}}
\index{readDataFile@{readDataFile}!file.h@{file.h}}
\subsubsection{\setlength{\rightskip}{0pt plus 5cm}struct {\bf dataset}$\ast$ read\-Data\-File (FILE $\ast$ {\em fh})}\label{file_8h_a4}


Reads the data file into a data set

\begin{Desc}
\item[Parameters:]
\begin{description}
\item[{\em fh}]Pointer to data file \end{description}
\end{Desc}
\begin{Desc}
\item[Returns:]Dataset containing data from file. \end{Desc}


Definition at line 126 of file file.c.