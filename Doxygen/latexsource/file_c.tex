\documentclass[11pt]{article}
\begin{document}
\begin{verbatim}

/*
 * file.c
 * Aquiring data from datafiles
 */

#include <stdio.h>
#include <stdlib.h>
#include <string.h>

#include "main.h"
#include "file.h"
#include "convert.h"

/* 
 * get[x|y|z]Num: return the number of x,y or z values
 * given the type of plot
 */
long getxNum (struct dataset *ds)
{
	return ds->X * ds->Y;
}

long getyNum (struct dataset *ds)
{
	return ds->X * ds->Y;
}

long getzNum(struct dataset *ds)
{
	return ds->X * ds->Y;
}

struct dataset *scaleDataSet(struct dataset *ds)
{
	double min, max;
	int i;

	/* Need to find Max and Min value regardless of axis */
	min = ds->x[0]; max = ds->x[0];
	for (i=0; i<getxNum(ds); i++) {
		if (ds->x[i] < min)
			min = ds->x[i];
		if (ds->x[i] > max)
			max = ds->x[i];
	}
	for (i=0; i<getyNum(ds); i++) {
		if (ds->y[i] < min)
			min = ds->y[i];
		if (ds->y[i] > max)
			max = ds->y[i];
	}
	for (i=0; i<getzNum(ds); i++) {
		if (ds->z[i] < min)
			min = ds->z[i];
		if (ds->z[i] > max)
			max = ds->z[i];
	}

	/* Scale all values so they lie in a unit cube */
	max = max - (max + min)/2; min = min - (max + min)/2;

	for (i=0; i<getxNum(ds); i++) {
		ds->x[i] = (ds->x[i] - (max + min)/2) * (1/max);
	}
	for (i=0; i<getyNum(ds); i++) {
		ds->y[i] = (ds->y[i] - (max + min)/2) * (1/max);
	}
	for (i=0; i<getzNum(ds); i++) {
		ds->z[i] = (ds->z[i] - (max + min)/2) * (1/max);
	}

	return ds;
}

void destroyDataSet(struct dataset *ds)
{
	if (ds == NULL)
		return;
	if (ds->x != NULL)
		free(ds->x);
	if (ds->y != NULL)
		free(ds->y);
	if (ds->z != NULL)
		free(ds->z);

	free(ds);
}

struct dataset *readDataFile(FILE *fh)
{
	struct dataset *ds;
	char buf[256];
	char tmp[256];
	int i,j;
	int line=0;
	int file_valid = 1;
	char type[5];
	int notpara = 0;
	double *t;
	double v;


	/* Initialise the data set structure */
	printf("Initialising the data structure...");
	ds = malloc(sizeof(struct dataset));
	ds->x = NULL;
	ds->y = NULL;
	ds->z = NULL;
	printf("done\n");


	/* Is this parametric data ? If not, flag it */
	fgets(buf, sizeof(buf) - 1, fh);
	if (strncmp(buf, "PARA", 4) == 0) {
		printf("Type PARA\n");
		notpara = 0;
	} else {
		notpara = 1;
		strncpy(type, buf, 4);
	}
	
	fgets(buf, sizeof(buf) - 1, fh);

	/* Line should be of form 'X,Y'; X,Y > 0 */
	i=0; j=0;
	while( i < strlen(buf) && buf[i] != ',' ) {
		tmp[j] = buf[i];
		i++; j++;
	}

	ds->X = atoi(tmp);
	i++; j=0;

	while( i < strlen(buf) ) {
		tmp[j] = buf[i];
		i++; j++;
	}

	ds->Y = atoi(tmp);

	/* Y is optional.. so still could be 0 */
	if (ds->X <= 0 || ds->Y < 0) {
		file_valid = 0;
		printf("readDataFile: File invalid; Line %d\n", line);
		return NULL;
	}

	/* Fix for KURV type, set Y=1 so room to store dataset */
	

	/* Allocate memory for dataset */
	printf("Allocate memory for dataset...");

	t = realloc(ds->x, getxNum(ds) * sizeof(double));
	printf("Num X values: %ld\n", getxNum(ds) );
	if (t == NULL) {
		destroyDataSet(ds);
		printf("readDataFile: failed while reallocating data");
		return NULL;
	}
	ds->x = t;

	t = realloc(ds->y, getyNum(ds) * sizeof(double));
	if (t == NULL) {
		destroyDataSet(ds);
		printf("readDataFile: failed while reallocating data");
		return NULL;
	}
	ds->y = t;

	t = realloc(ds->z, getzNum(ds) * sizeof(double));
	if (t == NULL) {
		destroyDataSet(ds);
		printf("readDataFile: failed while reallocating data");
		return NULL;
	}
	ds->z = t;

	printf("done\n");

	/* Convert dataset to PARA if required */
	if (notpara == 1) { 
		convertToPara(type, ds);
	}		

	
	/* x values */
	for (i=0; i< getxNum(ds); i++) {
		fgets(buf, sizeof(buf) - 1, fh);
		line++;
		v = strtod(buf, NULL);
		ds->x[i] = v;
	}
	/* y values */
	for (i=0; i< getyNum(ds); i++) {
		fgets(buf, sizeof(buf) - 1, fh);
		line++;
		v = strtod(buf, NULL);
		ds->y[i] = v;
	}
	/* z values */
	for (i=0; i< getzNum(ds); i++) {
		fgets(buf, sizeof(buf) - 1, fh);
		v = strtod(buf, NULL);
		ds->z[i] = v;
	}

	/* fclose(fh); */

	/* Scale all the values */
	ds = scaleDataSet(ds);

	return ds;
}

\end{verbatim}
\end{document}
